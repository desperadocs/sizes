% format: LaTeX2e
%   guide.tex
%------------------------
% $Author$
% $Date$
% $Revision$
% $HeadURL$
% $Id$
% User guide for the program: sizes.c

\documentclass[letterpaper]{article}
\pagestyle{headings}
\input boxedeps.tex
\input boxedeps.cfg
\HideDisplacementBoxes

\textwidth  =     6.5 in
\textheight =     8.0 in
\oddsidemargin =  0.0 in
\evensidemargin = \oddsidemargin

\def\tm{~(TM) }%%% how ==> ? {\protect{$^\mbox{TM}$}}
\newsavebox{\fminibox}
\newlength{\fminilength}
\newenvironment{fminipage}[1][\linewidth]
 {\setlength{\fminilength}{#1}%
    \addtolength{\fminilength}{-2\fboxsep}%
    \addtolength{\fminilength}{-2\fboxrule}%
    \begin{lrbox}{\fminibox}\begin{minipage}{\fminilength}}
 {\end{minipage}\end{lrbox}\noindent\fbox{\usebox{\fminibox}}}

\renewcommand\today{\number\day\space\ifcase\month\or
  January\or February\or March\or April\or May\or June\or
  July\or August\or September\or October\or November\or December\fi
  \space\number\year}

\author{Pete R. Jemian}
\title{User Guide for \protect\texttt{sizes}: 
       A Small-Angle Scattering Analysis Program}
\date{out-of-date: 1985}

\begin{document}

\maketitle

{\scriptsize 
 Portions of this document are from the 
 documentation supplied with the code \texttt{MAXE} 
 (documentation by Ian Culverwell, UKAEA-Harwell, 23 February 
 1987) and with its modifications called \texttt{MAXE2} 
 (modifications by Andrew Allen, UKAEA-Harwell, 19 July 1989) 
 and \texttt{MaxSas} (modifications by Pete Jemian, 
 1991-1995).
}

\section{Copyright}

 Copyright \copyright 1995, Late-Nite\tm Software

{\scriptsize   Permission to use, copy, and distribute this 
  software and its documentation for any purpose is 
  hereby granted without fee, provided that the above 
  copyright notice appear in all copies and that both
  that copyright notice and this permission notice 
  appear in supporting documentation, and that the 
  name of Late-Nite\tm Software not be used in 
  advertising or publicity pertaining to distribution 
  of the software without specific, written prior 
  permission.
  
  Permission to modify this software and 
  its documentation is not granted without express, 
  written permission for each and every instance.
  Late-Nite (tm) Software makes no representations 
  about the suitability of this software for any 
  purpose.  It is provided ``as is'' without express or 
  implied warranty.

  Portions of this code are, or have been adapted, 
  from the book ``Numerical Recipes in C.''

  \begin{center}\begin{verbatim}
   LATE-NITE (TM) SOFTWARE MAKES NO REPRESENTATIONS 
   ABOUT THE SUITABILITY OF THIS SOFTWARE FOR ANY 
   PURPOSE.  IT IS PROVIDED "AS IS" WITHOUT EXPRESS OR 
   IMPLIED WARRANTY.  LATE-NITE (TM) SOFTWARE DISCLAIMS 
   ALL WARRANTIES WITH REGARD TO THIS SOFTWARE, 
   INCLUDING ALL IMPLIED WARRANTIES OF MERCHANTABILITY 
   AND FITNESS FOR A PARTICULAR PURPOSE.  IN NO EVENT 
   SHALL LATE-NITE (TM) SOFTWARE BE LIABLE FOR ANY 
   SPECIAL, INDIRECT OR CONSEQUENTIAL DAMAGES OR ANY 
   DAMAGES WHATSOEVER RESULTING FROM LOSS OF USE, DATA 
   OR PROFITS, WHETHER IN AN ACTION OF CONTRACT, 
   NEGLIGENCE OR OTHER TORTUOUS ACTION, ARISING OUT OF 
   OR IN CONNECTION WITH THE USE OR PERFORMANCE OF THIS 
   SOFTWARE.
  \end{verbatim}\end{center}
}

  If you wish to make improvements to this code, 
  please contact the author at 
  \texttt{jemian@anl.gov} and you will be counted 
  among the blessed whose camels spit dates.  
  Those who 
  wish to violate this copyright, or its spirit, 
  will be considered lower than the ancestors of 
  the parasites infesting the toes of dung flies 
  and will be treated appropriately.


\tableofcontents

\section{Introduction}


\section{Theory}

\subsection{Small-Angle Scattering}

\subsubsection{Wavelength Smearing}

\subsubsection{Slit Smearing}

\subsection{Form Factor Choices}

\subsubsection{spheres}

\subsubsection{ellipsoids}

\subsubsection{globules}

The model for globules was developed by AJ Allen.

\subsubsection{rod globules}

The model for rod-shaped globules was developed by AJ Allen.

\subsubsection{disk globules}

The model for disk-shaped globules was developed by AJ Allen.

\subsection{Lagrange Undetermined Multipliers}

\subsubsection{Maximum Entropy}

\subsubsection{Regularization}

\section{Operation}

The program is divided into several distinct phases: 
input, editing, preparation, solution, and reporting.

\subsection{input}

First, the program reads the input command file and 
stores all the parameters.  As a cross-check for the 
investigator, the program then prints out all the 
parameters it has interpreted from the input command 
file.  If you are having trouble with the program, 
check this output first, to see if it matches what 
you have told the program to do.

Memory space is then 
allocated for the output size distribution.
Next the program reads 
the input data file in two passes.  The first pass is 
to count the number of data lines in the file.  After 
the first pass, memory is allocated for the arrays 
necessary to hold the input data and other information
required in the analysis.  A second pass through the 
data file is used to read the input data.

\subsection{editing}

After the input data has been read, the input data is 
edited to retain only the data points within the 
specified $Q$ range and to convert the data to units 
of absolute cross-section per unit solid angle per 
steradian ($I_j, \, \mbox{cm}^{-1} = \mathtt{fac} \times
(i_j - \mathtt{bkg})$).

\subsection{preparation}

Preparation involves the calculation of the 
\emph{grid} matrix for each combination of $Q_j$ and 
$r_i$.  This step can take some time for some shape 
models, such as ellipsoids of revolution measured 
with both wavelength-smearing and slit-smearing.

Presently there is only one choice of shape function 
(ellipsoids of revolution with possible wavelength 
and slit-length smearing)
but others are planned.  The investigator selects which shape 
function with the \texttt{shapeModel} parameter in 
the input data file.

\subsection{solution}

After the preceding steps are complete, the matrix 
problem can be submitted for solution.  The investigator 
selects which analysis method to use via the 
\texttt{analysisType} parameter in the input command 
file.

Each of the analysis methods will provide periodic 
updates as to their progress.  It is possible that an 
analysis method may not converge to a solution.  This 
is more likely with the MaxEnt method.  In that case, 
the program will not continue to the \emph{reporting} step.

\subsection{reporting}

After a successful completion of the \emph{analysis} 
step, the program will report the output size 
distribution and fit to the input data.  Finally, the 
program will quit.  (Note: on the Macintosh version 
of the program, the program will wait for the investigator to 
press [return] before quitting.)

\section{Examples}

There are several example files included with the 
distribution to aid in the use of the program.  These 
are:
\begin{itemize}
  \item \verb|model2.cmd|: Analysis of a synthesized 
   broad size distribution.
  \item \verb|ps-dsm.cmd|: Analysis of 460~nm 
   polystyrene spheres with a narrow distribution.
  \item \verb|ps-smr.cmd|: Analysis of 460~nm 
   polystyrene spheres with a narrow distribution 
   when the input data has not been corrected for 
   slit-smearing collimation effects.
  \item \verb|test.cmd|: Analysis of a synthesized 
     bimodal size distribution.
\end{itemize}

\subsection{Synthetic Bimodal $f(D)$ Distribution}
{\texttt{test.cmd}}

\subsection{Synthetic Broad Distribution}
{\texttt{model2.cmd}}

\subsection{Dow 4600~\AA\ Polystyrene Spheres}

A sample of polystyrene spheres from Dow (batch 
xxxxx), reported diameter 4600~\AA, 
was prepared by placing several drops of the 
spheres, suspended in liquid, onto scotch tape on a 
sample holder and then drying the liquid off.  The 
resultant sample was a densely packed cake of the 
spheres.  The scattering was measured on the 
double-crystal diffractometer at beam line X23A3 of 
the National Synchrotron Light Source, Brookhaven 
National Laboratory.  Since the thickness of the 
sample was indeterminate, it was not possible to 
place the scattered intensity from this sample on a scale of 
absolute cross-section per unit volume per steradian.

\subsubsection{Slit-smeared SAXS data}

After removal of the measured underlying 
rocking curve of the Si (111) optics, the data was 
analyzed by the program \texttt{sizes} for a 
distribution of scatterers using the regularization 
method giving \texttt{ps-smr.cmd} as the input command 
file.  The data recorded by the double-crystal 
diffractometer has been modified by the effect of 
slit-length collimation smearing, an effect which 
tends to smooth out the regular oscillations in the 
scattered intensity that arise from a low 
polydispersity in the size distribution.

The MaxEnt method has difficulty reconstructing the 
size distribution from this data set.  The 
oscillations are of small amplitude on the data set 
so that the MaxEnt procedure overshoots the solution 
and does not appear to recover.  This could be due to 
the specific MaxEnt search engine used.  A 
replacement of this engine may improve this method.  
However, the regularization method is capable of 
producing a solution that both fits the data and 
seems credible with what is already known about the 
sample.  The fit to the smeared data is shown in 
Fig.~\ref{f:ps-smr.fit}.  There is still some 
systematic difference in the fit, probably due to a 
distribution of sizes not probed in the analysis.
The volume fraction size 
distribution, $f(D)$, is shown in 
Fig.~\ref{f:ps-smr.f-D}.  It appears that the peak in 
the size distribution is at 4770~\AA.
%xxxxxxxxxxxxxxxxxxxxxxxxxxxxxxxxxxxxxxx
%xxxxxxxxxxxxxxxxxxxxxxxxxxxxxxxxxxxxxxx  What size?
%xxxxxxxxxxxxxxxxxxxxxxxxxxxxxxxxxxxxxxx

\begin{figure}[tbp]
 %\centerline{\BoxedEPSF{}}
 \caption{Small-angle scattering data from 
    Dow polystyrene spheres.  Analysis method 
    accounts for the slit-smearing of the data.  
    $\circ$: original data;
    $---$: fitted data}%%%%%% gotta find a better straight line symbol
 \protect\label{f:ps-smr.fit}
\end{figure}

\begin{figure}[tbp]
 %\centerline{\BoxedEPSF{}}
 \caption{Size distribution of Dow polystyrene spheres
    calculated by the regularization method.  
    4600~\AA\ was the reported size.}
 \protect\label{f:ps-smr.f-D}
\end{figure}

\subsubsection{Desmeared SAXS data}
{\texttt{ps-dsm.cmd}}


\section{File Formats}

All files used by the program, both input and output, 
are in the \emph{text} format native to the computer 
system being used.  For most computers, that format 
is ASCII.  This keeps the files transportable between 
systems and easily inspected by the investigator.

Be certain, when modifying a file with a word 
processing program that ``text'' or ``text with line 
breaks'' is chosen when writing the file.

Many of the files allow the use of comment lines.  In 
most cases, these comment lines begin with a \verb|#| 
character.  The correct form of comment line is 
described with each type of file below.

\subsection{Input Command File}

The program uses an input command file to select the 
various parameters to be used in the analysis.  The 
input command file is also useful as a reference for the 
documentation of a particular analysis.  The name of 
the input command file is passed to the program on 
the command line (or pseudo-command line window on 
the Macintosh version of the program).  For the input 
command file shown (\texttt{test.cmd}), that line 
would look like:
\begin{verbatim}
   sizes test.cmd
\end{verbatim}

Note: The input command file can have any name.
An example of an input command file is shown 
in Fig.~\ref{f:test.cmd}.
Use of the extension \verb|.cmd| is purely arbitrary.
\begin{figure}[tbp]
\begin{center}
\begin{fminipage}[0.8\linewidth]
\small\begin{verbatim}                test : Project Name  (only 1st item is read)
            test.sas : SAS file, contains columns: Q  i  esd
    1e-08        100 : qMin qMax, 1/A  (1.0e-8 to 100 means all data)
                 100 : rhosq  : scattering contrast, 10^20 1/cm^-4
                   1 : fac    :   I = fac * ( i - bkg )
                   1 : err    : ESD = fac * err * esd
                 0.1 : bkg    :   I = fac * ( i - bkg )
                   1 : shapeModel  (1=spheroids, no others yet)
                   1 : Aspect Ratio
                   0 : slitLength, 1/A
                   1 : Bin Type (1=Lin, 0=Log)
                  40 : nRadii
       25        900 : dMin dMax, A
                   1 : n, in N(D)*V^n
              1.0e-6 : defaultDistLevel  (MaxEnt only)
                  32 : IterMax
                  10 : iPlot (ignored for now)
              0.0002 : dLambda/Lambda
                   0 : analysisType (1=MaxEnt, 0=regularization)
\end{verbatim}
\end{fminipage}
\end{center}
 \caption{\texttt{test.cmd}: Example of an input command file.}
 \protect\label{f:test.cmd}
\end{figure}

In general, each line of the input command file 
contains two types of information: parameter(s) used 
by the program and comments for the investigator.  
The first one or two fields on each line are for the 
parameters, the rest of the line is ignored by the 
program.  White space (tabs or spaces) is used to 
separate each field.  The lines of the input command 
file must appear in the order shown or the program 
will perform with unpredictable results, usually 
crashing in the process.

\begin{description}
  \item[Project name] This parameter will be used to 
     name the output files of the program.  The 
     output files have the form of 
     \verb|<project>.<ext>|.  The program will assign 
     the file extensions.  (Investigators with 
     DOS-style 8.3 file names can maintain 
     compatibility by choosing project names with 8 
     or fewer characters.)  It is possible to prepend 
     a path name onto the project name if output is 
     to be located on another path.  An example 
     of this on a UNIX system might be: \verb|../results/test|.
  \item[SAS file] This is the name of the input data 
     file.  If the file is not in the same directory 
     or folder as the input command file, then this 
     must be either a link (or alias) to the file or 
     include the path name to the file.  An example 
     of this on a UNIX system might be: \verb|../data/test.sas|.
  \item[qMin qMax] ($\mbox\AA^{-1}$)  Analyze all data points from the 
   input data file between qMin and qMax.
   (1.0e-8 to 100 means ``analyze all data with 
   positive $Q$.'')
  \item[rhosq] ($10^{20} \, \mbox{cm}^{-4}$)  Scattering 
   contrast ${|\Delta\rho|}^2$; difference 
   in the scattering length densities of the 
   scatterer and its surroundings.
  \item[fac] $I_j = \mathtt{fac} \times ( i_j - \mathtt{bkg})$, 
      Factor used to 
   scale the intensities from the input data file to 
   absolute cross section ($\mbox{cm}^2$) per unit 
   volume ($\mbox{cm}^3$)  per steradian ($\mbox{sr}$).
  \item[err] $\sigma_j = \mathtt{fac} \times 
      \mathtt{err} \times esd_j$  
      Additional factor used to 
   scale the errors from the input data file to 
   absolute cross section ($\mbox{cm}^2$) per unit 
   volume ($\mbox{cm}^3$)  per steradian ($\mbox{sr}$).
  \item[bkg] $I_j = \mathtt{fac} \times ( i_j - \mathtt{bkg})$, 
   Constant 
   background to be subtracted from
   the intensities from the input data file.
  \item[shapeModel]  Selects which shape function to 
   use. (1=spheroids, no others yet)
  \item[Aspect Ratio]  Aspect ratio, $\beta$, of the 
   prototypical scatterer of radii: $r \times r 
   \times \beta r$
  \item[slitLength] ($l_0, \, \mbox\AA^{-1}$) For input data 
   that is affected by slit smearing, this is the 
   slit length to use in the analysis.  The 
   slit-length distribution function is a rectangular 
   profile.  If there is no slit-length smearing, set 
   this parameter to 0.  (No slit-smearing will be 
   calculated when $l_0 < Q_1$.
  \item[Bin Type] Selects whether the bins in the 
   size distribution will be evenly spaced on a 
   linear (1) or a logarithmic (2) scale.
  \item[nRadii]  Number of bins in the size 
   distribution.
  \item[dMin dMax] ($\mbox\AA$) Minimum and maximum 
   diameters in the size distribution.
  \item[n] Volume weighting exponent of the size 
   distribution, as in $N(D) \, V^n(D)$.  
   $n=0$ for a number distribution,
   $n=1$ for a volume fraction distribution, or
   $n=2$ for an ``intensity'' distribution.
   Use of values other than those indicated here, 
   such as non-integer values, is not recommended.
   The output of the program is likely to be 
   ill-behaved in such circumstances.
  \item[defaultDistLevel] (MaxEnt only)  The MaxEnt 
   analysis method requires the investigator to make 
   an initial estimate of the magnitude of the size 
   distribution below which the result is 
   meaningless.  This estimate is in terms of amount,
   not in terms of size.
  \item[IterMax]  Maximum number of iterations 
   allowed before the program will give up and report 
   ``no solution.''
  \item[iPlot]  (ignored for now)
  \item[dLambda/Lambda]  $\Delta\lambda/\lambda$.  
   Half-width of incident wavelength distribution.
   Presently, the wavelength distribution is a 
   triangular function.  This can be easily changed 
   to a Gaussian function by editing the compiler 
   variable \texttt{WL\_DIST\_TYPE} in file 
   \texttt{grid.h} and recompiling the code.  For a 
   Gaussian distribution, $\Delta\lambda/\lambda$ is 
   the $\sigma$ of the distribution.
  \item[analysisType] Selects the analysis method to 
   be used.  Presently, the choices are regularization(0)
   or MaxEnt (1).
\end{description}


\subsection{Input Data File}

The input data file contains the data to be analyzed.
The format of the input data file allows for \emph{comment 
lines} and \emph{data lines} in the file.  

A comment line is 
any line that begins with a ``\verb|#|.''  Any 
comment lines in the input data file will be written 
to the output fit and size distribution files.  Also 
to the output log file, if it is generated.  It is 
possible to remove individual data points from the input data 
file by commenting them out.  In the output files, 
these removed data points will appear with the other 
program comments.  Up to 4000 characters 
in all comment lines will be retained by the program.

The $j$-th data 
line consists of three numbers, all on the same 
line.  These numbers should be $Q_j \quad I_j \quad 
\sigma_j$, where $Q_j$, the magnitude of the $j$-th scattering 
vector, is in units of $\mbox{\AA}^{-1}$.  $I_j$ and 
$\sigma_j$ are the $j$-th scattered intensity and estimated 
standard deviation of $I_j$, respectively.  The program 
has an internal conversion routine to convert these 
into units of absolute cross-section per unit 
volume $\mbox{cm}^{-1}$
if needed.  The numbers on the data lines can be 
separated by either tabs or spaces.  However, all 
three numbers must be on the same line since the 
program counts the number of data lines in order to 
determine how many data points are in the file.

The $Q_j$ do not need to be regularly spaced.  
They do, however, need to be positive.  The $I_j$ do 
not need to be all positive, even after the removal 
of the background value.  The program will almost 
certainly have difficulty if many of the $I_j$ are 
negative.

Here's an example of the first few lines of an input 
data file: \par
\begin{center}
\begin{fminipage}[0.6\linewidth]
\small\begin{verbatim}
# test.sas
#
# Q,1/A         SAS,1/cm    esd,1/cm
4.0157139E-03 3497.473 90.72816 
4.5408653E-03 3340.003 84.95314 
5.0095972E-03 3322.474 79.63133 
...
\end{verbatim}
\end{fminipage}
\end{center}

\subsection{Output Fit File}

Once the program has determined that the convergence 
criteria have been met, then a data file containing
the fit to the selected size distribution will be 
created.  If a file with the name 
\texttt{<project>.fit} already exists, it will be 
overwritten with the new fit.

The first line in the output fit file describes the 
project name, the analysis method, and the time/date 
stamp of the completion of the analysis.  It is 
written to the file as a comment line with the format:
\begin{verbatim}
    fprintf (path, 
        "# project: %s, %s analyzed: %s\n",
        ProjectName, analysisMethod, timeDate);
\end{verbatim}

If there were any comment lines in the input data 
file, they will be copied to the output fit file as 
comments.  The data is then written in the file with the format:
\begin{verbatim}
   fprintf (path, 
     "%lg\t%lg\t%lg\t%lg\t%lf\n", 
      Q[j], SAS[j], dSAS[j], fit[j], (SAS[j]-fit[j])/dSAS[j]
   );
\end{verbatim}
The format means that the fields are separated by 
``tabs,'' a format accepted as the field 
delimiter by many plotting and spreadsheet programs.
These variables have the meaning:
\begin{itemize}
  \item \verb|Q[j]|: The independent variables, $Q_j$, from 
     the input data file
  \item \verb|SAS[j]|: The small-angle scattering intensity, 
     $I(Q_j)$ (1/cm) calculated from 
     the input data and the applied scaling factor 
     and background.  $I_j = \mathtt{fac} \times 
     (i_j - \mathtt{bkg})$ is the conversion function.
  \item \verb|dSAS[j]|: The estimated standard deviation of 
     the small-angle scattering intensity, 
     $\sigma_j$ (1/cm) calculated from 
     the input data and the applied scaling factor.
     $\sigma_j = \mathtt{fac} \times \mathtt{err} 
     \times \mathtt{esd}_j$ is the conversion function.
  \item \verb|fit[j]|: The intensity ($\hat I_j$) 
     re-calculated from the 
     fitted size distribution.
  \item \verb|(SAS[j]-fit[j])/dSAS[j]|: The standardized 
     residuals ($z_j \equiv (I_j - \hat I_j)/\sigma_j$) 
     of the fit.  For a good fit, $z_j$ should 
     be scattered randomly between $-1$ and $+1$.
\end{itemize}

All comment lines in the output fit file have a 
``\verb|#|'' as the first character.  All other 
lines in the file are data lines.

Here's an example of the first few lines of an output 
fit file: \par
\begin{center}
\begin{fminipage}[0.8\linewidth]
\small\begin{verbatim}
# project: test, regularization analyzed: 95.11.02, 07:53:49
# data file comment:
# test.sas
#
# Q,1/A         SAS,1/cm    esd,1/cm
#
# Q,1/A SAS,1/cm dSAS,1/cm fit,1/cm z
0.00401571 3497.37 90.7282 3610.6 -1.247956
0.00454087 3339.9 84.9531 3392.34 -0.617278
0.0050096 3322.37 79.6313 3189.1 1.673591
...
\end{verbatim}
\end{fminipage}
\end{center}

\subsection{Output Size Distribution File}

The output size distribution file may take one of 
three names depending on the input parameter $n$, 
which weights the volume term in the size 
distribution ($N(D) \times V^n(D)$).  

Table~\ref{t:size-file-extensions} shows the various 
possibilities.  The most common weighting is the 
volume fraction distribution, $f(D)$, where $n=1$.
The number distribution, $N(D)$, is used to make 
comparisons with other particle counting methods, 
such as transmission electron microscopy.  The 
intensity distribution, $i(D)$, is rarely used.
\begin{table}[tbp]
 \caption{Possible names for the output size 
 distribution file, dependent on the input parameter 
 $n$.}
 \begin{center}
   \begin{tabular}{|c|c|c|c|}
  \hline
  $n$ & $N(D) \times V^n(D)$ & file name & description  \\
  \hline
  0 & N(D) & \verb|<project>.N-D| & number distribution  \\
  \hline
  1 & f(D) & \verb|<project>.f-D| & volume fraction distribution  \\
  \hline
  2 & i(D) & \verb|<project>.i-D| & intensity distribution  \\
  \hline
   \end{tabular}
 \end{center}
 \protect\label{t:size-file-extensions}
\end{table}

Here's an example of the first few lines of an output 
size distribution file: \par
\begin{center}
\begin{fminipage}[0.8\linewidth]
\small\begin{verbatim}
# project: test, analyzed: 95.11.02, 07:53:49
# data file comment:
# test.sas
#
# Q,1/A         SAS,1/cm    esd,1/cm
#
# D,A f,1/A
25 3.6243e-06
47.4359 2.95622e-06
69.8718 -5.37022e-06
...
\end{verbatim}
\end{fminipage}
\end{center}

\subsection{Output Log File}

A log file can be obtained for some versions of the 
program.  On the Macintosh, a special feature of the 
Symantec C compiler is used to simultaneously log any 
screen output into a file.  The name of the file so 
produced is \texttt{<project>.log}.  On UNIX systems, the same 
log file can be obtained by directing the program output 
into a file with a UNIX command such as
\begin{verbatim}
  sizes input.cmd > project.log
\end{verbatim}
where \texttt{input.cmd} is the name of the input 
command file and \texttt{project.log} is the name of 
the output log file.  However, use of this UNIX 
command will redirect the output into the file; no 
program output will appear on screen.
Fig.~\ref{f:test.log} gives an example of an output 
log file.
\begin{figure}[tbp]
\begin{center}
\begin{fminipage}[0.8\linewidth]
\small\begin{verbatim}

 General Size Distribution Analysis of SAS data: 95.11.02, 07:53:40

Review of input parameters for analysis:
                test : ProjectName
            test.sas : SASfile
    1e-08        100 : qMin  qMax, 1/A
                 100 : RhoSq
                   1 : fac
                   1 : err
                 0.1 : Bkg
                   1 : shapeModel
                   1 : Aspect
                   0 : slitLength, 1/A
                   1 : Bin type (1=Lin, 0=Log)
                  40 : num bins
       25        900 : Dmin & Dmax
                   1 : n, in N(D)*V^n
               1e-06 : defaultDistLevel
                  32 : IterMax
                  10 : iPlot (ignored for now)
              0.0002 : dE_E
                   0 : analysisType (0=regularization, 1=MaxEnt)

Reading SAS from file: test.sas ...91 points were read.
91 points remained after Q-range editing.
file comments:
# test.sas
#
# Q,1/A         SAS,1/cm    esd,1/cm
Preparing the G(h,r) matrix ...(95.11.02, 07:53:42)
Beginning the solution routine ... (95.11.02, 07:53:44)
       log10(a)             ChiSqr                  S     Q=ChiSqr + a*S
              0            59.4836         0.00410335            59.4877
             50            78936.3        1.36203e-81            78936.3
             25            78936.3        1.36203e-31            78936.3
           12.5            10313.3        2.04245e-09            16772.1
           6.25            79.9625        7.13782e-06            92.6555
          9.375            1205.18        1.96006e-07            1669.98
         7.8125            153.436        2.96259e-06            345.821
        7.03125            89.9127        4.99894e-06            143.632
\end{verbatim}
\end{fminipage}
\end{center}
 \caption{\texttt{test.log}: Example output log file.}
 \protect\label{f:test.log}
\end{figure}


\subsection{Interim Shape Function Files}

Files containing the calculated shape functions, $S(u)$, can
be obtained for some of the chosen shape functions.

For the shape function modeling ellipsoids, the interim
shape function files may be obtained by defining the
compiler variable:
\begin{verbatim}
   #define WRITE_INTERMEDIATES      /* useful when debugging */
\end{verbatim} 
in the file \texttt{grid.h}.\ and rebuilding the 
executable file.

The data lines of the file are written with the format:
\begin{verbatim}
  fprintf (path, "%lg\t%lg\n", x[i], y[i]);
\end{verbatim}
in the routine: \texttt{WriteFile}.

The files written by the ellipsoid model are:
\begin{description}
   \item[\texttt{P\_lambda.dat}]: wavelength distribution $p(\lambda)$
   \item[\texttt{wave-\emph{i}.dat}]: $i$-th wavelength 
             smearing integrand $p(\lambda) \, 
             S_\lambda(u_i \, \bar{\lambda}/\lambda)$
   \item[\texttt{S\_s.dat}]: Form factor for spheres $S_w(u)$
   \item[\texttt{S\_e.dat}]: Form factor for 
             ellipsoids $S_e(u)$, if applicable
   \item[\texttt{S\_w.dat}]: Wavelength-smeared form factor for 
             ellipsoids $S_{e,\lambda}(u)$, if applicable
\end{description}

\section{Strategies}

Sometimes, it is helpful to have a little 
coaching\dots\quad There are many parameters that can be 
adjusted in the input command file.  Complementary 
investigations or other \emph{a priori} knowledge about 
the sample can be beneficial in expediting the analysis.
This section will attempt to describe how to use the 
various parameters to advantage while homing in on an 
acceptable result.

It is unlikely that subtle changes in the specified
$\Delta\lambda/\lambda$ will affect the result unless the 
polydispersity of the sample is quite low (\emph{i.e.}, 
the size distribution is sharply peaked).

The \texttt{rhosq} and \texttt{fac} parameters will 
only affect the magnitude 
of the output size distribution and will not affect the 
shape of the distribution or the convergence of the chosen 
analytical method.

\subsection{The New Sample}

For the first-time analysis of SAS data from a sample, one 
may not be certain of the best values for all of the 
parameters.  Experience is a good guide.  So is \emph{a 
priori} knowledge or complementary investigations of the 
sample.

The convergence tests of the \emph{MaxEnt} analysis are 
more stringent than for the \emph{regularization} method.
While the latter allows the resultant size distribution to 
have negative values, the regularization result obtained 
can be used as a guide to 
better estimate some of the input parameters.

In most cases, there is no penalty for increasing the 
range of $D$ and the number of points in the 
distribution.  Be warned that increasing the number of points
will increase the computation time proportionately.  
However, an inadequate number of points may fail to 
reproduce the measured scattering correctly, even if the 
other parameters are correct.  (Verify this on the 
\texttt{test} data set by setting the number of $D$ to 20 or 
less.)

If the chosen $D$ range spans more than about a 
decade, then it is more advisable to use the \texttt{log} 
spacing for $D_i$ values, since it likely \emph{maps} more closely 
to the spacing of the input $Q_j$ values.

\subsection{$Q$-range}

Choose the input $Q$ range from the data based on your 
judgement.  It may not be possible to analyze simultaneously 
4 decades of 
$Q$ data (you should be so fortunate to have this type 
of data).  Some of the smallest $Q$ values may be affected 
by penumbra (\emph{shadowing}) of the beam stop.

The program will perform unpredictably and likely crash 
for data with $Q_j \leq 0$.  Only use positive $Q_j$-values!

Don't waste time including a lot of flat background or 
negative intensity points in the analysis.  This does not 
improve the result.

\subsection{The \texttt{err} Parameter}

The \texttt{err} parameter is the most easily abused 
parameter.  One is tempted to increase it slightly in 
order to force convergence of a particular analysis.  The 
purpose of \texttt{err} is to correct for 
inadequately estimated errors in the input data.  Remember 
that \texttt{err} is applied as a constant factor to \emph{all} the 
input $\mathtt{esd}_j$.
 
\emph{Use caution!}  Be sure that the final value of 
\texttt{err} chosen has physical significance.  
When tracking down better estimates of the other parameters, 
adjusting \texttt{err} (usually by $\times 3$ to $\times 10$) 
can be useful to force convergence.  Remembering that one 
convergence criterion for MaxEnt and for regularization is 
that $\chi^2 = M$, where $M$ is the number of input data 
points, one can force convergence to a different point in 
the entropy or smoothness of the size distribution.  Just 
make sure that the $\sigma_j$ in the final analysis have physical 
significance.

\subsection{Background}

In some input data sets, the sample background may not 
be a constant value or it may not be known at all.  If it 
is not flat, investigate the reasons for this!  Is it due 
to another source of small-angle scattering, such as 
clustering?  Is it due to an overlap in the $Q$-range with 
the interatomic structure factor?  Is the 
shaped-background function due to an instrumental effect?
All of these possibilities should be handled \emph{before} 
the data is analyzed with this program or artifacts will 
certainly appear in the resultant size distribution.

If the background is not known, several methods exist to 
estimate it.  One, which only applies when a 
well-established Porod region exists, 
is the \emph{Porod} plot wherein the 
data is plotted as $I\,Q^4$ \emph{vs}. $Q^4$ ($Q^3$ for 
slit-smeared data).  According to Porod (reference Porod 
in Glatter and Kratky here), the slope of the plot, ignoring the 
first few data points, is the instrumental background.

Often, the background can be estimated visually by 
examination of a plot of $\log(I_j)$ \emph{vs}. $\log(Q_j)$.
At the highest $Q_j$ values, the data may begin to go flat.
Use your judgement.

\subsection{Linear- \emph{vs}. Log-spaced $D$ Bins}

As stated above, the $\log$-spaced $D$ bins will likely 
\emph{map} more evenly onto the input $Q_j$ values if the 
ratio $D_{max}/D_{min} > 30$ or so.  This is because the 
scattering at higher angles tends to over-sample the 
smaller size scatterers and under-sample the larger sizes
because most step-scanning (or 
radial binning) algorithms tend to use constant $\Delta Q$ 
intervals.  The $\log$ spacing provides more $D$ bins at 
the smaller sizes and fewer at larger sizes.

Do not become very exercised about this difference as it 
is truly second order for the most common input data sets, 
where $Q_{max}/Q_{min} < 100$.  Also, it may not be 
important for most size distributions that are clearly 
peaked within the size range investigated.  However, for 
the SAS data arising from a very broad size distribution 
(nearly fractal, some may say), the $\log$-spacing is the 
correct choice.

\subsection{The \texttt{defaultDistLevel} Parameter}

For a better understanding of the purpose of the
the \texttt{defaultDistLevel} parameter used in
the \texttt{MaxEnt} analysis method, it is best to read
Skilling and Bryan's paper.  If the order of magnitude of 
the value is not known, switch to the regularization 
method and obtain an output size distribution.  Make your 
first estimate of \texttt{defaultDistLevel} as 5\% (or so) of the 
peak of the distribution.  Switch back to the MaxEnt 
method and proceed.  Convergence will probably be 
obtained.  Observe that by adjusting 
\texttt{defaultDistLevel}, the structure of the 
distribution at (usually) the 
smaller sizes will be affected most.

This procedure is best to follow when switching the 
volume weighting parameter, $n$, as in $N(D) \times V^n(D)$
since the magnitudes of $N(D)$, $f(D)$, and $i(D)$ are 
very different.

\subsection{The \texttt{iterMax} Parameter}

Increasing the \texttt{iterMax} parameter is useful for 
some troublesome analyses.  For most analyses, convergence 
should be obtained in 15-25 iterations of the selected 
method.  For \texttt{regularization}, the number of 
iterations is directly proportional to the number of 
significant bits of precision for a double-precision 
floating point number.  This is because the search for the 
Lagrangian parameter $\alpha$ is done by bisection.  
Usually 32 is a reasonable number, even though the 
precision of a \texttt{double} may be 64 or greater.

Troublesome analyses occur more often with \texttt{MaxEnt} 
analyses, when the $\chi^2=M$ criterion has been met but 
the parallelism criterion between the $\vec{\chi^2}$ and $\vec s$ 
vectors has not been reached.  At this point, the MaxEnt 
search engine
is shifting values in the distribution between $D_i$ bins 
to make these two vectors parallel within 5\% (compiler 
variable \verb|TEST_LIMIT| in file \texttt{maxent.c}).

Other troublesome analyses happen when the input data has 
some small structure with respect to the $\sigma_j$.  
These are just the hard facts of life; you may need better data.

%\section{Troubleshooting}


\section{References}

\begin{enumerate}
\item
        J. Skilling and R.K. Bryan; \emph{Mon Not R Astr Soc} 
        \textbf{211} (1984) 111 - 124.
\item
        J.A. Potton, G.J. Daniell, and B.D. Rainford; Proc. Workshop 
           on Neutron Scattering Data Analysis, Rutherford Appleton 
           Laboratory, UK, 1986;
           ed. M.W. Johnson, \emph{IOP Conference Series} 
           \textbf{81} (1986) 81 - 86, Institute
           of Physics, Bristol, UK.
\item
        I.D. Culverwell and G.P. Clarke; Proc. Workshop 
           on Neutron Scattering Data Analysis, Rutherford Appleton 
           Laboratory, UK, 1986;
           ed. M.W. Johnson, \emph{IOP Conference Series} 
           \textbf{81} (1986) 87 - 96, Institute
           of Physics, Bristol, UK.
\item
        J.A. Potton, G.J. Daniell, \& B.D. Rainford; 
           \emph{J Appl Cryst} \textbf{21}
           (1988a) 663 - 668.
\item
        J.A. Potton, G.J. Daniell, \& B.D. Rainford;
           \emph{J Appl Cryst} \textbf{21}
           (1988b) 891 - 897.
\item
        L.C. Roess \& C.G. Shull; 
           \emph{J Appl Phys} \textbf{18} (1947) 308-313.
\end{enumerate}

\end{document}
